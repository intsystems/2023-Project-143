\documentclass[17pt]{ article}

\usepackage[utf8]{inputenc}
\usepackage[english, russian]{babel}
\usepackage[T1]{fontenc}

\begin{document}

\begin{center}
    Рецензия на рукопись.

    Дистилляция моделей и данных. 

    Баринов Н.
\end{center}

\subsection*{Статья:}
\begin{itemize}

\item Introduction написан хорошо, мне нравится. 

\item В пункте 3.1 дистилляция данных мне непонятно, $\theta_t^* := \theta_t^*$, т.к. там написано, что инициализируем случаным значением параметры модели-студента, нужно поправить обозначения.

\item В пункте 3.2 Дистилляция моделей совсем не понятна форма записи функции потерь, наверное, стоит пояснить её.

\item В пункте 3.3 не написана новая функция потерь. 

\item В пункте 4.1 в таблице пропущено значение для ConvNet, хотя эксперименты для неё проведены.

\item Вывода пока нет

\end{itemize}

\subsection*{Графики:}
Стоит сохранить картинки, а не делать скриншоты, это сильно портит качество картинок.

Можно зарегестрироваться на kaggle, чтобы проделать более масштабные эксперименты с большим количеством эпох, 30 часов вычислительной мощности в неделю это позволяет. Я делал там, это удобно.

\subsection*{Код:}
\begin{itemize}
    \item Модуль networks не загружен на github.
    \item В статье идет речь об экспериментах только на CIFAR10, внутри есть ещё код для экспериментов на MNIST, может быть ещё не добавили или уже убрали.
    \item Код понятный, читаемый, эксперимент воспроизводится, нужно убрать импорт из networks, т.к. модели уже были добавлены в файл, а не импортируется из модуля.
\end{itemize}

Рецензент:
\newline Крейнин М.В.

\end{document}